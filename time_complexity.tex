\section{Time complexity $subsums_{\mathrm{graph star}}$ algorithm}


\begin{algorithm}[h]
    \caption{Check if a GSP subsumes a $Q_s$ ($subsums_{\mathrm{graph star}}$)}\label{alg:containmentTree}
    \begin{algorithmic}[1]
       \algsetup{linenosize=\tiny}
       \scriptsize
 
       \REQUIRE  $Q_{star}$, $Q_{starG_i}$, $Q_s = Q_{s\mathrm{body}} \cup Q_{s\mathrm{unions}}$, $Eval_{star}$
       \ENSURE \TRUE $ $ or \FALSE $ $ whether the root of a graph star pattern $Q_{star}$ subsumes a shape.
 
       \IF{$M(S_{star}(Q_{star})) \in Eval_{star}$}
          \RETURN $M(S_{star}(Q_{star}))$
       \ENDIF 
 
       \FORALL{$tp \in Q_{star}$} % O(n_star)
          \IF{\NOT $match(tp, Q_{s\mathrm{body}})$}
             \STATE $hasOnePath \gets $ \FALSE
             \FORALL{$q_{us} \in Q_{s\mathrm{unions}}$}
                \IF{$subsums_{\mathrm{graph star}}(Q_{star}, Q_{starG_i}, q_{us}, Eval_{star})$}
                   \STATE $hasOnePath \gets $ \TRUE
                \ENDIF
             \ENDFOR
             \IF{\NOT $hasOnePath$}
             \STATE $Eval_{star} \gets Eval_{star} \cup (M(S_{star}(Q_{star})) \mapsto \mathrm{false})$
                \RETURN \FALSE
             \ENDIF
          \ELSE
             %eval nested
             \IF{$O(tp) \in  S_{star}(Q_{starG_i})$}
               \STATE $Eval_{star} \gets Eval_{star} \cup M(S_{star}(Q_{star})) \mapsto \mathrm{true}$
                \IF{\NOT $subsums_{\mathrm{graph star}}(Q_{star_{O(tp)}} \in Q_{starG_i}, Q_{starG_i}, Q_s, Eval_{star})$}
                  \STATE $Eval_{star} \gets Eval_{star} \cup (M(S_{star}(Q_{star})) \mapsto \mathrm{false})$
                   \RETURN \FALSE
                \ENDIF
             \ENDIF
             %
          \ENDIF
       \ENDFOR
 
       %\STATE $Eval_{star} \gets Eval_{star} \cup M(S_{star}(Q_{star})) \mapsto \mathrm{true}$
       \RETURN \TRUE
    \end{algorithmic}
 \end{algorithm}

We aim to prove that the algorithm $subsums_{\mathrm{graph star}}$ terminates in polynomial time and, more precisely, that its time complexity is
\begin{equation}
O\left(n_{sunion} \cdot n_{tp}^{2}\right),
\end{equation}
where $n_{sunion}$ is the number of union branches in the query $Q_s$ and $n_{tp}$ is the total number of triple patterns in the input graph star pattern $Q_{starG_i}$.

\paragraph{Preliminaries}
We assume that the \texttt{match} function operates in constant time, returning \texttt{true} if and only if the domain of matched triples of a triple pattern in the query is a superset of that of the first argument triple pattern, that is, if the second pattern is as general as or more general than the first.
It has to be noted that in the definition of $Q_{starG_i}$, there cannot be shared triple patterns in $Q_{star}$.
In this algorithm we suppose that $Eval_{star}$ is passed by reference.

\paragraph{Worst-case per $Q_{star}$}
Let $n_{qstar}$ denote the number of distinct $Q_{star} \in Q_{starG_i}$.
For each node $Q_{star}$, the algorithm iterates over its triple patterns (line 4 to 25), resulting in a time complexity of $O(n_{tp_{Q_{star}}})$.
For each triple pattern that does not match the shape body ($Q_{s\mathrm{body}}$), the algorithm iterates over all union branches in $Q_{sunion}$ (line 7 to 11), making at most $n_{sunion}$ recursive calls.
The algorithm traverse the $q_{us}$ graph however each $q_{us} \in Q_{sunion}$ cannot contain a Union Graph Pattern (UGP), and is therefore always of the form $Q_s = Q_{s\mathrm{body}}$, making this 
branch (line 5 to 15) of the algorithm after a first execution not the worst case with a complexity of $O(n_{tp_{Q_{star}}}^{2} \cdot n_{sunion})$.

After the first execution, the worst-case scenario becomes one in which the triple patterns matches a pattern in $Q_{s\mathrm{body}}$, and the condition $O(tp) \in S_{star}(Q_{starG_i})$ holds (line 17 to 22).
In such cases, the algorithm recursively explores the corresponding partial Graph Star Pattern (GSP) by executing $subsums_{\mathrm{graph star}}$, following a graph traversal paradigm.


\paragraph{Tree Traversal Argument}
Although $Q_{starG_i}$ is a graph, the algorithm avoids cycles due to "caching", once a node $Q_{star}$ is evaluated, its result is stored in $Eval_{star}$.
Therefore, each node is visited at most once, and the overall traversal is equivalent to a tree traversal of size $n_{qstar} \leq |Q_{starG_i}|$.

\paragraph{Total Complexity}
Let $n_{tp_{Q_{star}}}$ denote the number of triple patterns in a particular $Q_{star}$, and let $n_{tp}$ be the total number of triple patterns across all nodes, so
$$
n_{tp} = \sum_{Q_{star} \in Q_{starG_i}} n_{tp_{Q_{star}}}
$$

Then the total number of recursive operations over all nodes is bounded by:
$$
O\left(\sum_{i=1}^{n_{qstar}} n_{sunion} \cdot n_{tp_{Q_{star_i}}}^{2}\right)
$$

$$
O\left(n_{sunion} \cdot \sum_{i=1}^{n_{qstar}} n_{tp_{Q_{star_i}}}^{2}\right)
$$

$$
O\left(n_{sunion} \cdot \left(n_{tp}^{2} - \sum_{i=1, j=1}^{n_{qstar}}  n_{tp_{Q_{star_i}}} \cdot  n_{tp_{Q_{star_j}}}\right)\right)
$$

$$
O\left(n_{sunion} \cdot n_{tp}^{2}\right)
$$

\section{Time Complexity of the $subsums_{\mathrm{Q}}$ Algorithm}

\begin{algorithm}[h]
    \caption{Check if a query $Q$ subsumes a $Q_s$ ($subsums_{\mathrm{Q}}$)}\label{alg:containmentQuery}
    \begin{algorithmic}
        \algsetup{linenosize=\tiny}
        \scriptsize

        \REQUIRE  $Q$, $Q_s$ and $Eval_{star}$
        \ENSURE \TRUE $ $ or \FALSE $ $ whether the root of a graph star pattern $Q$ subsumes a shape.

        \FORALL{$Q_{starG_s} \in Q$}
            \FORALL{$Q_{star} \in Q_{starG_s}$}
                \IF{$subsums_{\mathrm{graph star}}(Q_{star}, Q_{starG_s}, \emptyset)$}
                    \RETURN \TRUE
                \ENDIF
            \ENDFOR
        \ENDFOR
        \RETURN \FALSE
    \end{algorithmic}
\end{algorithm}

The time complexity of the $subsums_{\mathrm{Q}}$ algorithm is straightforward to derive, as it consists of iterating over the Graph Pattern Structures (GPS) of the query and applying the $subsums_{\mathrm{graph star}}$ algorithm. Thus, the time complexity is given by:
\begin{equation}
O\left(n_{sunion} \cdot n_{tp_{max}}^{2} \cdot n_{Q_{star_{max}}} \cdot n_{Q_{starG}}\right),
\end{equation}

Where, $n_{tp_{max}}$ is the maximum number of triple patterns in any $Q_{starG}$,
$n_{Q_{star_{max}}}$ is the maximum number of $Q_{star}$ patterns in any $Q_{starG}$,
$n_{Q_{starG}}$ is the number of Graph Pattern Structures (GPS) in the query $Q$.\footnote{
    Sharing the evaluation results $Eval_{star}$ from $subsums_{\mathrm{graph star}}$ could reduce execution time and potentially the algorithm's complexity.
}
